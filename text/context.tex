\section{The Importance of Context}
\label{sec:context}

While the translation we have presented in \sect\ref{sec:formalization}
provides a boilerplate to transform \ears into \ltl, the rules given in
\tab\ref{fig:translation_ears_ltl} are naive regarding the interaction between
the complete set of requirements in an \ears specification. For example, if one
take into consideration the transformation rule for \emph{Event-Driven} requirements, the \ltl formlula
corresponding to those will only state that in the moment immediately after the
\emph{trigger} is received, the \emph{response} will be given. However,
the requirement states nothing about the life of the system after
that moment. It is reasonable to ask questions such as:

\begin{enumerate}
  \item after a \emph{trigger} is received, will the \emph{response}
  keep produced by the system for some period?
  \label{q1}
  \item if so, until when?
  \label{q2}
\end{enumerate}

While it is challenging to come up with a generic answer for these questions,
we will attempt to describe a solution for the case in which such a specification is
used for code synthesis, as is done in our \earsctrl tool.

As mentioned in \sect\ref{sec:introduction}, \earsctrl was built to synthesize
software controllers directly from requirements written in \ears. Software
controllers receive input from \emph{sensors} and produce output on
\emph{actuators}. By orchestrating those signals a software controller
controls the operation of a machine, such as the system of motors presented in
\fig\ref{fig:reqs_motor}.

Coming back to questions \ref{q1} and \ref{q2} above, let us attempt to answer
then by analysing a concrete example based on our motor controller case study.
Part of \textsf{Req1} states that when the \emph{start button} is pressed then
the \emph{oil motor} will start. It is assumed that the motor will continue to
be on until it is shutdown\footnote{Note that we assume that $\text{\emph{start
motor oil}} = \neg(\text{\emph{stop motor oil}})$. In \earsctrl this is
explicitely expressed in a glossary accompanying the requirements in
\fig\ref{fig:reqs_motor}.}, which can only happen if \textsf{Req4} applies
(meaning the \emph{stop button} has been pressed).
However, this is information that can only be deduced by taking \textsf{Req4}
into consideration.

In order to incorporate this contextual information in an \ltl specification,
static analysis of the \ears requirements becomes necessary. Such a static
analysis is embodied in the algorithm in \fig\ref{algo:staticanalysis}. The
algorithm calculates which triggers exist in other \ears requirements that
should be taken into consideration during the translation and accumulates them
in the $untilClause$ set. The body of the algorithm goes through all other
requirements in the specification and checks whether the responses overlap, in
which case their \emph{triggers} are added to the $untilClause$ set. Such
a static analysis needs to be done for all \ears templates in
\tab\ref{fig:translation_ears_ltl}, except for the \emph{Ubiquitous} template --
in which case the response should in any case always be present in the system.

\begin{figure}
\begin{algorithmic}
	\STATE $untilClause = \emptyset$
	\STATE $currentReq = \text{requirement being transformed}$
	\STATE $contextReq = \text{first requirement different from } currentReq$
    \WHILE{not all context requirements have been
    processed}
    	\IF{$currentReq.responses \cap contextReq.responses$}
    	        \STATE $untilClause = untilClause \cup {contextReq.trigger}$
    	\ENDIF
	\ENDWHILE
	\caption{Algorithm to calculate contextual information}
	\label{algo:staticanalysis}
\end{algorithmic}
\end{figure}

If we now revisit the translation of \textsf{Req1} to include contextual
information calculated by the static analysis algorithm, we can now produce the
following \ltl formula:

\begin{multline*} 
\mathbf{G}(\text{\emph{start button pressed}} \rightarrow\\
(\mathbf{X}(\text{\emph{start oil motor}}\land \text{\emph{start 10 sec
timer}})\\ \mathbf{U} \text{\emph{ stop button pressed}}))
\end{multline*}

The $\mathbf{U}$ operator in the formula enforces that the \emph{start motor on}
actuator will be left on $\mathbf{U}$ntil the \emph{stop button} is pressed.
Note that the \ltl formula contains only one proposition after the until operator.
This is in general not the case for the algorithm \fig\ref{algo:staticanalysis},
as more than one proposition can be accumulated in the $untilClause$ set -- in
which case the propositions after the until operator appear disjuncted in
the corresponding generated \ltl formula.

% 
% \begin{itemize}
%   \item $UntilClause = \emptyset$
%   \item $forall req $
% \end{itemize}