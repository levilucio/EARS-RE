\section{Introduction and Problem Statement}

\ears~\cite{EARS09} has been designed at Rolls-Royce having in mind helping
requirements engineers to specify requirements that are as unambiguous, complete
and objective as possible -- while being written in natural language.
Technically architecting and writing software is an expensive task.
Misunderstandings in communicating requirements to architects and developers
amount to using precious software development resources in an unfruitful manner,
leading to project overbudgeting and/or failure~\cite{chaos:2014}. Mavin \etal
have shown with their work that, by using a simple set of template english
sentences for specifying requirements, ambiguity, partiality and subjectivity
can be significantly reduced in textual requirements, even for large
specifications~\cite{EARS10,EARS16}.

A byproduct of having requirements written in a simple subset of natural
language is that automation becomes more possible. The regularity of \ears lends
itself well to mathematical treatment and, given the wealth of research in
formal methods, it is only reasonable to investigate how \ears specifications
can be turned into fully precise formal specifications. Such formal
specifications can then be used for purposes such as \emph{code synthesis},
\emph{formal verification} or \emph{test generation}.

The work we present here is motivated by the IETS3 project\footnote{IETS3 was
funded by the German Federal Ministry of Education and Research under code
01IS15037A/B} ran recently by our company on the construction of languages for
requirements engineering. With IETS3 we have been able to synthesise code
directly from \ears specifications. To that end, we have transformed \ears into
Linear Temporal Logic (LTL), a mathematical formalism that allows expressing
temporal dependencies between the states of a system. This transformation is
part of the \earsctrl environment~\cite{earsctrlProcess} that allows
synthesizing code directly from requirements expressed in
\ears~\cite{LucioRCA16,LucioRAM17}.

In this paper we will elaborate on the transformation between \ears 

\begin{itemize} 
  \item Motivate why it is important for formalize EARS
  \item Talk about context, IETS3 project
  \item Specifically EARS for controllers, domain where EARS shines 
  \item Formal language of choice: LTL
\end{itemize}