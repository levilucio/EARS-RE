The Easy Approach to Requirements Specification (\ears)
EARS\cite{earsctrlProcess}\cite{LucioRCA16}\cite{LucioRAM17} has
been designed primarily as a set of templates to assist requirements engineers
in writing software requirements that are clear and understandable.
Its target are thus requirements engineers software architects and developers.
Due to the minimalistic nature of the english sentences that make up an EARS
specification, it is reasonable to expect that automated tasks can be performed
on EARS specification, among which verification and code synthesis. Given
English cannot be directly understood by machines without some degree of
ambiguity, EARS requirements can only by automatically processed if they are
translated in advance into formal specifications. In this short paper we explore
how a translation from EARS into Linear Temporal Logic can be implemented as
well as what challenges are raised by moving from structured natural language
into a formal language.