\section{Open Issues and Discussion}
The results obtained by utilizing EARs in the context of IETS3 are
promising. In particular, EARs being utilized, 1) as a structured
requirements writing language as opposed to writing in natural language and, 2)
to do automatic code synthesis from EARs specifications. However, using EARs in
our context have also led us to have a list of interesting open issues and
observations.
The following is a non-exhaustive list of open issues of EARs
being utilized in the context of IETS3,
\begin{itemize}
  \item Limited scope of translation controller generation. E.g. dealing with
  parameters is not taken care of. (Is this an extensibility issue? can we
  solve it by extending the language with parameters? is this an observation?)
  \item Covering the gap between informal and formal -- how far can we get
  without losing implicit information. Most of the organizations write
  requirements in an unstructured/natural language. one of our
  observations is that it is difficult for the companies to view immediate
  benefits of adopting a structured/template language like EARs. For that
  purpose, we would like to investigate further on how to close the gap between
  natural/informal requirements and structured/template language like EARs so
  that there is a minimal loss of information.
  \item How generalizable is our approach? At present, we have applied our
  approach for automatic code synthesis in the controller domain. We would like to investigate if our approach can
  be generalized enough and can be utilized in other domains (e.g., writing test
  cases).
  \item Model checking is the next step. We would further like to investigate
  how EARs can be helpful in performing the model checking.
  \item Order is often important. It can be deduced from the context, but should
  there be an implicit order? At present, we have written requirements using
  EARs for controller generation purpose where the writing in a particular order might not that
  important. However, it would be interesting to investigate further as to how
  to prioritize/order the requirements when written in EARs for other domains?
  FOr instance, this might be helpful in order to run test suites attached to
  high-priority requirements.
  \item Requirements are often not independent, we have observed during our
  investigations that there is not an explicit way of having/managing
  dependencies between requirements when written in EARs.
  \item At present, EARs allow only discrete events (at any instance in time
  as binary ON/OFF), how to have EARs for continuous (real-time) events in time? 
\end{itemize}